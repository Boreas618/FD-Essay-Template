\documentclass[UTF8,AutoFakeBold]{ctexbook}
\usepackage{xeCJK}
\usepackage{ctex}
\usepackage{geometry}
\usepackage{times}
\usepackage{graphicx}
\usepackage{hyperref}
\usepackage{amsmath}
\usepackage{setspace}
\usepackage{caption}
\usepackage{titlesec}
\usepackage{indentfirst}
\usepackage{booktabs}
\usepackage{mathptmx}

\geometry{a4paper,scale=1, left=3.18cm, right=3.18cm, top=2.54cm, bottom=2.54cm}

\titleformat{\section}[block]{\normalfont\bfseries}{\thesection}{1em}{}
\titleformat{\subsection}[block]{\normalfont\bfseries}{\thesubsection}{1em}{}
\titleformat{\subsubsection}[block]{\normalfont\bfseries}{\thesubsubsection}{1em}{}
\titlespacing*{\section}{0pt}{0pt}{0pt}
\titlespacing*{\subsection}{0pt}{0pt}{0pt}
\titlespacing*{\subsubsection}{0pt}{0pt}{0pt}

\setlength{\parindent}{2em}

\renewcommand{\thefootnote}{\textcircled{\raisebox{-0.9pt}{\arabic{footnote}}}}

\setCJKmainfont{SimSun.ttc}

\begin{document}
\hypersetup{hidelinks} 
\pagestyle{empty}

\begin{spacing}{1.0}
\begin{center}
    \zihao{3} \textbf{主标题}
    
    \textbf{}
    
    \zihao{4} \textbf{——副标题(若无副标题,请删除)}\par
\end{center}
\end{spacing}

\setlength{\baselineskip}{20pt}
\zihao{-4} 

\noindent\textbf{【摘要】}

一般不超过300字,摘要应建立在对论文进行总结的基础上,对文章内容、结构和观点进行的概括,供读者清楚了解、论文提出的问题、采用的证据和得出的结论。字体为宋体小四号字。摘要中不应出现“本文”“本研究”等表述。

\noindent \textbf{【关键词】} 

关键词控制在3—5个之间,一般不超过3个。字体为宋体小四号字。摘要与关键词之间不空行,关键词与正文之间空1行。

\noindent \textbf{【正文】} 

\setlength{\baselineskip}{20pt}

\section*{一、关于正文内容的说明}

\subsection*{(一)引文注释}

注释是对所创造的名词术语的解释或对文中引文出处的说明。
格式:采用“脚注”形式。具体项目和格式如下:

\subsubsection*{1. 专著、论文集、学位论文和报告类:}
[序号]主要作者(多个作者间用逗号“、”隔开):文献题名,出版地:出版者,出版年,起止页码。 \footnote[1]{周振莆:《周易译注》,北京:中华书局,1991年,第20-25页。} \footnote[2]{John Dewey, Democracy and Education: An Introduction to Philosophy of Education, New York: Macmillan (1966 paperback edition), 1916, p.25.}

\subsubsection*{2. 译著:}
[序号]主要作者(多个作者间用逗号“、”隔开):文献题名,译者,出版地:出版者,出版年,起止页码。\footnote[3]{奥兰多·费吉斯:《娜塔莎之舞:俄罗斯文化史》,郭丹杰、曾小楚译,成都:四川人民出版社,2018年,第4页。} 

\subsubsection*{3. 期刊文章:}
[序号]主要作者(多个作者间用逗号“、”隔开):文献题名,刊名,年卷(或期),起止页码。 \footnote[4]{何龄修:《读顾城<南明史>》,《中国史研究》,1998年第3期,第163-173页。} \footnote[5]{S.Kanter, Z.Gamson \& H.London, “The Implementation of General Education: Some Early Findings,” The Journal of General Education, 1991(40), pp.119-127.}

\subsubsection*{4. 电子文献:}
[序号]作者,题名,文献出处,发表或更新日期[引用日期],获取和访问路径。 \footnote[6]{傅刚,《大风沙过后的思考》,《北京青年报》,2000-04-12[2002-03-06],http://www.bjyouth.com.cn/Bqb/20000412/gb/ 4216\%5ed0412b.htm.}

\subsection*{(二)图片的格式规范要求}

\subsubsection*{1. 形式:}
各种框图、流程图,须通过word 的绘图工具或使用Visio等专业的制图软件进行制图;

\subsubsection*{2. 位置:}
插入文章的各类型图都应居中,并按排版需要调整大小,但图片宽度不应超过正文文本宽度;

\subsubsection*{3. 图题:}
图题在图片下方,包含图序和图名,图序与图名之间应留1 个同类字符的空隙;居中,字体黑体加粗,字号比正文小一号。图序使用阿拉伯数字依序编排。

\begin{figure}[h]
\centering
\includegraphics[width=0.5\textwidth]{example-image}
\caption*{\heiti\bfseries\zihao{-5} 图1 图片标题}
\end{figure}

\subsection*{(三)表格的格式规范要求}

\subsubsection*{1. 形式:}
论文中的各种表须使用在word 文档中插入表格的形式进行表格的绘制;

\subsubsection*{2. 位置:}
表格应居中放置,大小适中,宽度不宜超过论文文本宽度

\subsubsection*{3. 表头:}
表头在表格上方,包含表序和表名,表序与表名之间应留1 个同类字符的空隙,居中放置,与表格顶线无间隔,字体黑体加粗,字号比正文小一号。表序使用阿拉伯数字依序编排;

\subsubsection*{4. 内容:}
在表中除了具体内容是因文章而异,表内文字字体、字号、数字书写规范等都应该统一。

\begin{table}[htbp]
\centering
\caption*{\heiti\bfseries\zihao{-5} 表1 表格标题} % 表头
\begin{tabular}{|c|c|c|} % 三列居中对齐
\hline
\textbf{列1} & \textbf{列2} & \textbf{列3} \\ \hline
内容1 & 内容2 & 内容3 \\ \hline
内容4 & 内容5 & 内容6 \\ \hline
\end{tabular}
\end{table}

\subsection*{(三)公式的格式规范要求}
公式统一空两格左对齐,公式采用Word自带公式编辑器输入,编号靠右侧,采用小四Times New Roman字体斜体。

\[
\int_{0}^{\infty} e^{-x^2} dx = \frac{\sqrt{\pi}}{2}
\]

\subsection*{(四)参考文献}
格式:采用“尾注”形式,在正文结束后,罗列在后。

说明:如有必要列举论文涉及的其他文献,则可在文末列出参考文献。具体项目和格式如下:

\subsubsection*{1. 专著、论文集、学位论文和报告类:}
[序号]主要作者(多个作者间用逗号“、”隔开):文献题名,出版地:出版者,出版年,起止页码~\cite{ref1}~\cite{ref2}。 

\subsubsection*{2. 译著:}
[序号]主要作者(多个作者间用逗号“、”隔开):文献题名,译者,出版地:出版者,出版年,起止页码~\cite{ref3}。

\subsubsection*{3. 期刊文章:}
[序号]主要作者(多个作者间用逗号“、”隔开):文献题名,刊名,年卷(或期),起止页码~\cite{ref4}~\cite{ref5}。 

\subsubsection*{4. 电子文献:}
[序号]作者,题名,文献出处,发表或更新日期[引用日期],获取和访问路径~\cite{ref6}。 


\let\cleardoublepage\clearpage

\begin{thebibliography}{99}
\bibitem{ref1} 周振莆:《周易译注》,北京:中华书局,1991年,第20-25页。
\bibitem{ref2} John Dewey, Democracy and Education: An Introduction to Philosophy of Education, New York: Macmillan (1966 paperback edition), 1916, p.25.
\bibitem{ref3} 奥兰多·费吉斯:《娜塔莎之舞:俄罗斯文化史》,郭丹杰、曾小楚译,成都:四川人民出版社,2018年,第4页。
\bibitem{ref4} 何龄修:《读顾城<南明史>》,《中国史研究》,1998年第3期,第163-173页。
\bibitem{ref5} S.Kanter, Z.Gamson \& H.London, “The Implementation of General Education: Some Early Findings,” The Journal of General Education, 1991(40), pp.119-127.
\bibitem{ref6} 傅刚,《大风沙过后的思考》,《北京青年报》,2000-04-12[2002-03-06],http://www.bjyouth.com.cn/Bqb/20000412/gb/4216%5ed0412b.htm.
\end{thebibliography}

\end{document}
